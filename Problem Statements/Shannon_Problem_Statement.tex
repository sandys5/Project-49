\documentclass[journal,10pt,onecolumn,draftclsnofoot,]{IEEEtran}
\usepackage[utf8]{inputenc}
\usepackage[pass, letterpaper]{geometry}
\usepackage[T1]{fontenc}
\begin{document}
\title{Problem Statement\\
    \large CS 461, Fall Term}
\author{\large Shannon Sandy}


\maketitle

\section{Abstract}
Our goal for this project is to create a 3D animated video of the Apollo 11 mission. The video will be 25 minutes and will accurately depict Apollo 11's flight from Earth to the moon and back. Since the video will need to be realistic, we will need to research aspects of the Apollo 11 flight like the trans-lunar injection, command module docking, and trans-Earth injection. In order for the audience to understand the different aspects of Apollo 11’s flight, we will need to display text within the video that explains what is currently happening. We may also have to include audio in the video so the audience is not watching a silent film.
\newpage
\section{Problem Definition and Proposed Solution}
Our problem is that we need to re-create the Apollo 11 mission with 3D graphics. The video must be 25 minutes long and accurately depicts all aspects of the mission. This means that we will have to animate all components of the Earth launch, trans-lunar injection, lunar flight, moon landing, moon launch, command module docking, trans-Earth injection, flight to Earth, and the Earth landing. In order to depict the flight realistically, we must implement physics into our 3D animation. 

Since this video will be for audiences to view at the Oregon Museum of Science and Industry, we will also need to ensure that a general audience will be able to understand what is happening during different aspects of the Apollo 11 flight. This might look like adding text to the video that explains what is happening during each aspect of the flight as well as how long it took the Apollo 11 to complete that phase. If we include text in the video, we will also have to include an audio recording of the text so the video can be accessible to general audiences. In order to include this information, we must research the Apollo 11 mission so we can have a deep understanding of the flight. The overall purpose of this video is to be able to view a recreation of the Apollo 11 mission from vantage points that allow the audience to appreciate the complexity of it.

The proposed solution to this problem is to research the flight of Apollo 11 so we know what we will have to animate in 3D graphics. We will also have to research the physics that affected the Apollo 11 flight so we can implement them into our animation. If for some reason we cannot implement physics for all aspects of the flight, we will use idealized motion instead. When creating the video, we will use 3D model, color, and texture data for as much of the video as we can. This data will be obtained from the client. For parts of the video where that data in unavailable, we will use 3D graphics and texture mapping to create convincing imitations. If needed, we will also add text and audio in order to elaborate on different aspects of Apollo 11's flight.

\section{Performance Metrics}
This project will be considered complete after the following criteria have been completed. The video must be close to 25 minutes in length and implement physics and 3D model, color, and texture data as much as possible in the video. We will ask the client more about this data and if they know where it can be found. The video must also accurately portray every aspect of the Apollo 11 flight. These aspects include the launch from Earth, trans-lunar injection, lunar flight, moon landing, moon launch, command module docking, trans-Earth injection, flight to Earth, and the Earth landing. The video must be accessible and obvious to general audiences that the Apollo 11 is being recreated. The video must also textually and audibly elaborated on lesser known aspects of the Apollo 11 flight like trans-lunar injection, command module docking, and trans-Earth injection. There may also be some background music added to the video so the video can be more entertaining to the audience.

All components of the video must be animated as well. We will deliver segments of the video to the client that depict a phase of the Apollo 11 flight in order to verify that the clip meets the client's requirements. The video will also have to be played at the Oregon Museum of Science and Industry in order to test if the museum's projector's software affects the speed of the animation. This will also give the client the opportunity to view the latest draft of the video and offer any last critiques. The video must also provide vantage points of the Apollo 11 flight that highlights the complexity and desolation of the mission. This means that the view of Apollo 11 must be close enough to see the details of what is happening but also far away enough so the audience can see how alone in space the crew was.
\end{document}
