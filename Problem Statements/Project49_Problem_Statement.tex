\documentclass[onecolumn, draftclsnofoot,10pt, compsoc]{IEEEtran}
\usepackage{graphicx}
\usepackage{url}
\usepackage{setspace}


\usepackage{geometry}
\geometry{margin=.75in}

\begin{document}

\title{Senior Capstone - Problem Statement\\
	\large 461 Senior Design - Fall 2018\\
	\large Jonathan Ropp, Dean Akin, Shannon Sandy\\
	\large October 16, 2018}

\maketitle

\begin{abstract}

July 2019 will mark the 50th anniversary of the Apollo 11 Moon Landing. To commemorate this anniversary, the Oregon Museum of Science and Industry (OMSI) has partnered with Oregon State University (OSU) to produce a twenty-five-minute 3D animation of the Apollo 11 mission to be displayed in the museum. This animation will encompass all aspects of the mission, from takeoff, to moon landing, to Earth landing, and every part in-between. This project is meant to give viewers an accurate depiction of the Apollo 11 mission so that they will be able to fully appreciate the scale and complexity of such an undertaking. This will require us to become familiar with all aspects of the mission, including trans-lunar injection, command module docking, trans-Earth injection, and more. The animation will be viewed by audiences of all ages and backgrounds, meaning that the end product needs to be made in a way that anyone can view and understand the stages and processes of the Apollo 11 Mission.

\end{abstract}

\newpage

\section{Problem Description}
Our problem is that we need to recreate the Apollo 11 mission using 3D graphics to capture the complexity and vastness of the mission in all its parts. We will turn the 3D animation into a 25 minute video that will need to include the Earth launch, trans-lunar injection, lunar flight, moon landing, moon launch, command module docking, trans-Earth injection, flight to Earth, and the Earth landing. We must also make the video in such a way that general audiences will enjoy viewing it as well as understand what is happening. This includes letting the viewers choose a variety of viewpoints from which to observe the mission. 

\section{Proposed Solution}
To create the best solution, research about the Apollo 11 mission is required. We will need to learn what our 3D models should look like, the course the Apollo 11 took to the moon and back, what every aspect of the flight looked like, how the physics of space affected the mission, and much more. To make the animation, we will primarily use a combination of C++ and OpenGL to produce the graphics. OpenGL allows us to insert 3D models into a scene, animate them, and view them from arbitrary viewpoints. This, combined with the logical programming power of C++ allows for high-end computer graphics managed by a powerful code base. Further into the project, we may decide in implement a solution in a different code base, such as the Unity game engine, to make our final product more accessible. Along with this, we will add audio to the animation for more viewing enjoyment, including soudns for the thrusters during takeoff, communications during the mission, and the mission-ending splashdown.
\newline
\newline
We will use 3D models that most accurately represent the Apollo 11 mission, using real images and textures when possible. When we cannot use real images, we will analyze the real mission closely and produce a solution that will be as close to the real thing as possible. A skybox with a starfield texture will have to be constructed around the entire scene for the video to show the galaxy. When animating the 3D models, we will use physics to represent as much movement as we can and using idealized motion in other scenarios for the most accurate representation possible. Things we will consider may include gravity, drag, thrust, escape velocity, and more. These will all play a part when launching the rocket, releasing the propulsion systems, landing on the moon, command module docking, Earth landing, etc. Once we have a prototype of the animation, we wont be able to directly show that in the OMSI planetarium. We will need to see if it is possible to access the Software Development Kit (SDK) that is used for the ten projectors in the planetarium. As long as it is possible, we will update our project so we can take full advantage of the equipment in the planetarium and hopefully have a video that looks even better than on a flat screen. 

\section{Performance Metrics}
We will know when the project is done when several benchmarks are finished. Ideally, we will be able to produce a twenty-five-minute long, 3D animation of all stages of the Apollo 11 mission that will be able to be shown at OMSI in their 10-projector planetarium. This would consist of the following scenes: Earth launch, trans-lunar injection, lunar flight, moon landing, moon launch, command module docking, trans-Earth injection, flight to the Earth, and Earth landing. All of these scenes will need to be smoothly strung together while also providing the viewer with the ability to view the scene from any vantage point they choose. Each scene will use as much physics for animation and actual texture data for the models as possible to accurately represent the mission. Also, we must design this project in a way that it can be played back over on any system or computer. While we will specifically make the animation for the planetarium, we also want animation to be robust enough so that it can be reused and added to in the future on multiple platforms.
\newline
\newline
Since this project will be on public display, it must be held to a certain standard of graphical quality. At a very minimum, a working prototype would be able to play from start to finish without any critical errors. Depending on what we decide with the client, we may also work on narration for the animation since it will be displayed in an educational setting. To make sure that we are meeting our client's expectations, we plan to send segments of the animation for feedback as we are able. Then, when we have a prototype, we can work with our client to attempt to display it in the planetarium as a first overall test. Lastly, and above all else, we want to align with OMSI’s mission statement: "Inspire curiosity through engaging science learning experiences, foster experimentation and the exchange of ideas, and stimulate informed action.” We will work to find any way that we can add effects, extra information, or different viewing settings so that all audiences can engage with the animation and take something meaningful from the presentation

\end{document}
