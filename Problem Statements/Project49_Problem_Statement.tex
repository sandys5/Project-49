\documentclass[onecolumn, draftclsnofoot,10pt, compsoc]{IEEEtran}
\usepackage{graphicx}
\usepackage{url}
\usepackage{setspace}


\usepackage{geometry}
\geometry{margin=.75in}

\begin{document}

\title{Senior Capstone - Problem Statement\\
	\large 461 Senior Design - Fall 2018\\
	\large Jonathan Ropp, Dean Akin, Shannon Sandy\\
	\large October 16, 2018}

\maketitle

\begin{abstract}

July 2019 will mark the 50th anniversary of the Apollo 11 Moon Landing. To commemorate this anniversary, the Oregon Museum of Science and Industry (OMSI) has partnered with Oregon State University (OSU) to produce a twenty-five-minute 3D animation of the Apollo 11 mission to be displayed in the museum. This animation will encompass all aspects of the mission, from takeoff, to moon landing, to Earth landing, and every part in-between. This project is meant to give viewers an accurate depiction of the Apollo 11 mission so that they will be able to fully appreciate the scale and complexity of such an undertaking. This will require us to become familiar with all aspects of the mission, including trans-lunar injection, command module docking, trans-Earth injection, and more. The animation will be viewed by audiences of all ages and backgrounds, meaning that the end product needs to be made in a way that anyone can view and understand the stages and processes of the Apollo 11 Mission.

\end{abstract}

\section{Problem Description}
Our problem is that we need to recreate the Apollo 11 mission using 3D graphics. The 25 minute video will need to include the Earth launch, trans-lunar injection, lunar flight, moon landing, moon launch, command module docking, trans-Earth injection, flight to Earth, and the Earth landing. We must also make the video in such a way that general audiences will enjoy viewing it as well as understand what is happening.
\section{Proposed Solution}
The proposed solution to this problem is to begin researching the Apollo 11 mission. This includes researching: what our 3D models should look like, the course the Apollo 11 took to the moon and back, what every aspect of the flight looked like, how the physics of space affected the mission, and what sounds the crew might have heard during the mission.

In order to make the video, our group could use a program like Unity. To make the video more enjoyable to watch, our group should add audio files to the video. These files would include the sounds of the rockets during launch and the discussions the crew had with the team on Earth. When making the video, we must implement physics where we can in order to be as realistic as possible.


\section{Metrics}


\end{document}
