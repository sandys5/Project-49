\documentclass[10pt,a4paper,oneside,onecolumn, english]{IEEEtran}
\usepackage[utf8]{inputenc}
\usepackage[english]{babel}
\usepackage{amsfonts}
\usepackage{amssymb}
\usepackage[left=0.75in,right=0.75in,top=0.75in,bottom=0.75in]{geometry}
\author{Dean A. Akin}
\title{CS461, Apollo 11 3D Animation, Fall 2018}
\begin{document}
\maketitle
\section{Abstract}
The Apollo 11 launch was a monumental success and definitely one of the biggest milestones that the human race has pulled off, but very little people remember it or even know of it. The Apollo 11 launch was the pinnacle of engineering and science at the time, and to get a craft into space, land a lunar module on the moon with two humans on board, and make a trip back to earth is not easy; yet it isn't talked about very much? This milestone was achieved 50 years ago, for it to be talked about by today's generation rarely happens, this is why OMSI would like to commemorate the launch by putting on a show within one of there planetariums. We must first discuss what OMSI wants specifically in there animation, this animation will be interactive such as being able to switch between views from the cockpit, Earth, etc. Once we have a fully working animation on a laptop, we must then figure out how to pipe it into the system that controls there planetarium, probably by there in-house SDK (Software Development Kit), and how we will know if we have achieved the story that we are trying to portray within this animation. 

\section{Description}
Our goal is to animate and plot the path of the Apollo 11 shuttle to the moon and back. This includes calculating the actual path of the Apollo 11 so that it looks realistic to the actual thing. We must also set the scene that is realistic to the conditions that the craft was launched in such as, have a sky box that has a texture of a star field/map that correctly portrays the stars and many other objects position at the time of the launch, have earth and the moon accurately placed as well as having accurate textures that portray the topology and geography of the two objects, and have an accurate model of the Apollo 11, and so on. 

\section{Solution}
Our solution is rather simple but tedious. Since we are working with Mike Bailey, the curriculum that he teaches applies heavily to how we are going to accomplish our goal, and it involves heavy use of OpenGL. In terms of plotting the path of the Apollo 11, plotting it is rather easy and only requires some basic fundamental physics and spacial awareness to interpolate it onto the screen, luckily we do not need to do any rocket science to recreate what they did on screen. Telling the story of the whole mission within the animation is the most important part, what and how we show things on screen matters. The obvious things that we must include are the Earth, Moon, and the Apollo 11; these three things must be portrayed properly in order the set the scene correctly, especially the placement as they didn't just aim the shuttle directly at the moon and launch it, they had to lead it and intersect the moon's orbit in order to land properly and not just crash and burn. Other things to include into the scene is the stars around us, this can be done by applying a star map onto a sky box so that the scene doesn't seem so empty. This show will also be interactive, and will allow the user to switch between different views such as the cockpit, Earth launch site, and so on. Not only that, the user will have access to a lot of meta data such as a side screen that shows the whole path of the Apollo 11 and the current location in terms of how far from the Moon and the Earth the Apollo 11 is, making this whole animation not only visually appealing but educational and truly commemorative. Other than doing the animation itself, our biggest hurtle is to pipe it into the planetariums display, which more than likely uses an in-house SDK to program the shows, but as of right now we do not have access to this and must deal with it when the time arises, this is why it is our biggest hurtle because it deals with a lot of the unknown.

\section{Performance Metrics}
We will know when the project is done when several benchmarks are finished. We must design this project in a way that it can be played back over on any system or computer, as to reuse it in the future, as well as make it easy to edit or build off of. Since this project has the possibility of being in a public space, it must be held to a certain standard of very little to no graphical errors. At a very minimum, a working prototype would be able to play from start to finish without any hiccups or errors including the interactivity of the animation, adding to the scene and making it look pretty is the easy part. In terms of a beta, all objects must be rendered within the scene correctly, this includes Earth, the Moon, Apollo 11, and the sky box(star map), and possibly more as the project moves on. The final version would be able to be displayed in the planetarium at OMSI fully, this includes the interpolation of the animation to the various screens as well as the programming of the show in their in-house SDK. If this cannot be accomplished, a full version of the animation that is feature complete would be the minimum to be considered finished.  
\end{document}
