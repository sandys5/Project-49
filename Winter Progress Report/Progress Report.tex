\documentclass[onecolumn, draftclsnofoot,10pt, compsoc]{IEEEtran}
\usepackage{graphicx}
\usepackage{url}
\usepackage{setspace}

\usepackage{geometry}
\geometry{textheight=9.5in, textwidth=7in}

% 1. Fill in these details
\def \CapstoneTeamName{		    The Apolloers}
\def \CapstoneTeamNumber{		49}
\def \GroupMemberOne{			Jonathan Ropp}
\def \GroupMemberTwo{			Shannon Sandy}
\def \GroupMemberThree{			Dean Akin}
\def \CapstoneProjectName{		Apollo 11 3D Animation}
\def \CapstoneSponsorCompany{	OMSI}
\def \CapstoneSponsorPersona{	Jim Todd}
\def \CapstoneSponsorPersonb{	Mike Bailey}

% 2. Uncomment the appropriate line below so that the document type works
\def \DocType{		
                %Problem Statement
				%Requirements Document Draft
				%Technology Review
				%Design Document
				Progress Report
				}
			
\newcommand{\NameSigPair}[1]{\par
\makebox[2.75in][r]{#1} \hfil 	\makebox[3.25in]{\makebox[2.25in]{\hrulefill} \hfill		\makebox[.75in]{\hrulefill}}
\par\vspace{-12pt} \textit{\tiny\noindent
\makebox[2.75in]{} \hfil		\makebox[3.25in]{\makebox[2.25in][r]{Signature} \hfill	\makebox[.75in][r]{Date}}}}
% 3. If the document is not to be signed, uncomment the RENEWcommand below
\renewcommand{\NameSigPair}[1]{#1}

%%%%%%%%%%%%%%%%%%%%%%%%%%%%%%%%%%%%%%%
\begin{document}
\begin{titlepage}
    \pagenumbering{gobble}
    \begin{singlespace}
        \hfill 
        % 4. If you have a logo, use this includegraphics command to put it on the coversheet.
        \includegraphics[height=4cm]{OSU_horizontal_2C_O_over_B.eps}   
        \par\vspace{.2in}
        \centering
        \scshape{
            \huge CS Capstone \DocType \par
            {\large\today}\par
            \vspace{.5in}
            \textbf{\Huge\CapstoneProjectName}\par
            \vfill
            {\large Prepared for}\par
            \Huge \CapstoneSponsorCompany\par
            \vspace{5pt}
            {\Large\NameSigPair{\CapstoneSponsorPersona}\par}
            {\Large\NameSigPair{\CapstoneSponsorPersonb}\par}
            {\large Prepared by }\par
            Group\CapstoneTeamNumber\par
            % 5. comment out the line below this one if you do not wish to name your team
            \CapstoneTeamName\par 
            \vspace{5pt}
            {\Large
                \NameSigPair{\GroupMemberOne}\par
                \NameSigPair{\GroupMemberTwo}\par
                \NameSigPair{\GroupMemberThree}\par
            }
            \vspace{20pt}
        }
        \begin{abstract}
        % 6. Fill in your abstract   
    
During this Winter term, our group has worked on creating a 3D animation for the Apollo 11 Moon Landing. We were finally able to meet Jim Todd in person and we changed some of our main goals of the project. The animation has reached Alpha functionality and is on track to be at Beta level by the end of the term. 

        \end{abstract}     
    \end{singlespace}
\end{titlepage}
\newpage
\pagenumbering{arabic}
\tableofcontents

% 7. uncomment this (if applicable). Consider adding a page break.
%\listoffigures
%\listoftables
\clearpage

% 8. now you write!

\section{Updated Goals}
On Thursday, January 3rd, our design group traveled to OMSI with Mike Bailey to meet with Jim Todd in person for the first time. We were treated to a full tour of the Harry C. Kendall Planetarium at OMSI and got to see the inner workings of their setup. We learned that the planetarium received a software upgrade about a year ago and now used DigitalSky: Dark Matter, a software produced by SkySkan, a provider of planetarium equipment and services. Jim showed us the Dark Matter interface on the main control computer and we got to see many different samples. This gave us a much better idea of how our project would look on a domed theater and how to best design the project to make the best use of the theater. 

At the time of our visit, our group's plan was to animate all portions of the Apollo 11 mission, from launch to splashdown. However, when talking with Jim, we changed course to instead focus more on what happened on the Moon's surface and instead use original news footage to depict what happened before and after. We have some creative freedom with specifics, but our current story board is now as follows: 

\begin{list}
    \item The animation will start with original video of the launch, show a flight path of the Saturn V rocket to the Moon, include audio snippets of relevant quotes, and then show the Lunar Lander descending on the surface of the Moon.
    \item The operator will now be able to interact with the animation, changing the view of the camera on the lunar surface. If no changes are made, the camera will follow a simulation of what Neil Armstrong did on the surface through his eyes. Key points will include: first steps on the surface, planting the American Flag, views of the Earth, and more. During this time, the operator will be able to take questions from the audience. 
    \item When the Operator is ready to end the animation, they will begin a scripted ending where Neil gets back in the Lunar Module and ascends from the surface, show the flight path again, show the Saturn V Rocket, audio snippets of relevant quotes, and then video of the splashdown on Earth, finishing with credits. 
\end{list}

Our group will implement this animation in OpenGL. Unfortunately, we recently discovered this will not work directly with the DigitalSky system at OMSI. Instead, we will need to translate our animation into a unique scripting language for use in the planetarium. This will be good because we will be able to make full use of the software's capabilities, but could result in some complications when trying to convert our project to that language. Mike Bailey has been, and is currently in contact with Jim Todd and a contact with SkySkan to find the best ways to complete this conversion. Converting our animation into the scripting language has been added as a stretch goal for our project.

\section{Current Progress}
As of today, February 18th, our team has created a program in OpenGL that has all objects in the scene, fully textured, including the Saturn V Rocket, Lunar Module, astronaut, Earth, Sun, Moon, landing zone, and a star map background. All objects have been placed in their starting positions and we have six viewpoints highlighting various features.

\section{To Do}
In order to have our beta completed, 

%An infinite point light has been placed at the Sun to illuminate the scene. 
%We have loaded one video...
%A general animated viewpoint has been made to survey the scene on the lunar surface
%(NUMBER) audio clips have been included 

%And what is left



\section{Problems and Solutions}
%Where to get good models
%How to include video
%How to design for expo AND OMSI


\section{Conclusion}


\end{document}
