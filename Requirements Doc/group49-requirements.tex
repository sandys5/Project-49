\documentclass[onecolumn, draftclsnofoot,10pt, compsoc]{IEEEtran}
\usepackage{graphicx}
\usepackage{url}
\usepackage{setspace}

\usepackage{geometry}
\geometry{textheight=9.5in, textwidth=7in}

% 1. Fill in these details
\def \CapstoneTeamName{		    The Apolloers}
\def \CapstoneTeamNumber{		49}
\def \GroupMemberOne{			Jonathan Ropp}
\def \GroupMemberTwo{			Shannon Sandy}
\def \GroupMemberThree{			Dean Akin}
\def \CapstoneProjectName{		Apollo 11 3D Animation}
\def \CapstoneSponsorCompany{	OMSI}
\def \CapstoneSponsorPerson{	Mike Bailey}

% 2. Uncomment the appropriate line below so that the document type works
\def \DocType{		
                %Problem Statement
				Requirements Document
				%Technology Review
				%Design Document
				%Progress Report
				}
			
\newcommand{\NameSigPair}[1]{\par
\makebox[2.75in][r]{#1} \hfil 	\makebox[3.25in]{\makebox[2.25in]{\hrulefill} \hfill		\makebox[.75in]{\hrulefill}}
\par\vspace{-12pt} \textit{\tiny\noindent
\makebox[2.75in]{} \hfil		\makebox[3.25in]{\makebox[2.25in][r]{Signature} \hfill	\makebox[.75in][r]{Date}}}}
% 3. If the document is not to be signed, uncomment the RENEWcommand below
%\renewcommand{\NameSigPair}[1]{#1}

%%%%%%%%%%%%%%%%%%%%%%%%%%%%%%%%%%%%%%%
\begin{document}
\begin{titlepage}
    \pagenumbering{gobble}
    \begin{singlespace}
        \hfill 
        % 4. If you have a logo, use this includegraphics command to put it on the coversheet.
        %\includegraphics[height=4cm]{CompanyLogo}   
        \par\vspace{.2in}
        \centering
        \scshape{
            \huge CS Capstone \DocType \par
            {\large\today}\par
            \vspace{.5in}
            \textbf{\Huge\CapstoneProjectName}\par
            \vfill
            {\large Prepared for}\par
            \Huge \CapstoneSponsorCompany\par
            \vspace{5pt}
            {\Large\NameSigPair{\CapstoneSponsorPerson}\par}
            {\large Prepared by }\par
            Group\CapstoneTeamNumber\par
            % 5. comment out the line below this one if you do not wish to name your team
            \CapstoneTeamName\par 
            \vspace{5pt}
            {\Large
                \NameSigPair{\GroupMemberOne}\par
                \NameSigPair{\GroupMemberTwo}\par
                \NameSigPair{\GroupMemberThree}\par
            }
            \vspace{20pt}
        }
        \begin{abstract}
        % 6. Fill in your abstract    
        	Our group, The Apolloers, is working with Mike Bailey to create a 3D animation about the Apollo 11 Moon Landing. This animation will be put on display in OMSI during the Summer of 2019 for the 50th anniversary of the Apollo 11 mission. All parts of the mission will be included, from Earth Launch to Earth landing, and all sections in between. This document breaks the project into requirements that we will use to guide our project through the development process. 
        \end{abstract}     
    \end{singlespace}
\end{titlepage}
\newpage
\pagenumbering{arabic}
\tableofcontents
% 7. uncomment this (if applicable). Consider adding a page break.
%\listoffigures
%\listoftables
\clearpage

% 8. now you write!
\section{Introduction}
Our group will be recreating the Apollo 11 mission in a 25 minute video using 3D graphics. This project is for the Oregon Museum of Science and Industry to display in their planetarium for the 50th anniversary of the Apollo 11 mission. This document contains the requirements for this project that our group and clients, Jim Todd and Mike Bailey, have agreed upon. These requirements act as a definitive list of features that our project will need to implement in order for it to be considered complete. 
    \subsection{Purpose}
    The purpose of this project is to educate the general public about the Apollo 11 mission as well as to commemorate the mission's 50th anniversary. Our goal is to create the video so that the audience at OMSI will appreciate the complexity of the mission while also being entertained. 
    \subsection{Scope}
    This project will be a 25 minute video whose target audience are the people attending the planetarium at OMSI. This includes school children on a field trip, people interested in space travel, and people who are attending the planetarium to be entertained. Since the planetarium at OMSI has a large audience of diverse people, the Apollo 11 recreation will need to be accessible, entertaining, as well as realistic in order to educate the audience without boring them.
    \subsection{Overview}
    For the video of the Apollo 11 to be considered complete the following features will need to be implemented: textured 3D objects, a variety of camera positions for the viewer, the flight path of Apollo 11, docking of the modules, physics, and audio files and transcriptions. This is a general overview of the requirements for this project. The System Requirements section will go into more detail explaining  
    \subsection{Definitions}
    \begin{itemize}
        \item SDK: The set of software tools that allows for applications of a certain software, hardware, computer system, operating system or similar development platform.
        \item OMSI: The Oregon Museum of Science and Industry and is located in Portland, OR.
    \end{itemize}
\section{Specific Requirements}
    \subsection{External Interfaces}
    At minimum, the animation will need to be viewed on some sort of computer display. Ideally, we will be able to gain access to OMSI's projector SDK so that we can display the animation in OMSI's planetarium through 10 projectors. There will also be audio alongside the animation, such as the sounds of the boosters, mission communications, and more (possibly captioned).
    \subsection{Functions}
    The 3D animation of the Apollo 11 Mission will include the entire flight path (launch to orbit, orbit to trans-lunar injection, to moon, to lunar orbit, to landing, to re-launch, to lunar orbit, to trans-Earth injection, to Earth, to splashdown). 3D objects will include the Earth, Moon, Lunar Module \textit{Eagle}, Saturn 5 rocket, Command Module, and others as we see fit. All of these sections will be smoothly animated together and be as scientifically accurate as possible.
    \subsection{Usability Requirements}
    Users will be able to view the animation from arbitrary viewpoints. This will be controlled by a computer mouse or some other input device. Any user with only basic computer knowledge should be able to change these viewpoints.
    \subsection{Performance Requirements}
    The animation will need to run at a steady frame rate throughout the whole mission to avoid breaking immersion. Also, there cannot be any errors when running the program; it will need to be robust enough to keep running even in edge-case environments.
\section{Verification}
    \subsection{External Interfaces}
    Minimally, we can make sure that we can view the animation from a computer display. Then, if we gain access to the projector SDK, we can attempt to view the animation in OMSI's planetarium and make sure that the animation scales to that size correctly. Also, we need to make sure the audio sounds good from a computer station, but if we present in the planetarium, we will need to make sure the audience is treated to the best audio the planetarium can provide. 
    \subsection{Functions}
    We will make sure to include all parts of the flight path in the animation and these scenes need to transition from one to the other without any noticeable change. We will have 3D objects for all notable objects in the animation (Earth, Moon, Saturn 5 Rocket, etc.). Ideally, we will use physics for all movement, but a reasonable goal would be to have at least half the movement be based on real physics and used idealized motion for the rest.
    \subsection{Usability Requirements}
    Allow a user from outside of our project group take control of the animation and make sure that they system is user-friendly.
    \subsection{Performance Requirements}
    On a suitable, mid-range computer, the animation should be able to run and keep a steady frame rate even when given extreme values for input. We will aim for a frame rate that varies less than +/- 5 frames per second.
\section{Gantt Chart}

\end{document}
