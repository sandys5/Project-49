\documentclass[onecolumn, draftclsnofoot,10pt, compsoc]{IEEEtran}
\usepackage{graphicx}
\usepackage{url}
\usepackage{setspace}

\usepackage{geometry}
\geometry{textheight=9.5in, textwidth=7in}

% 1. Fill in these details
\def \CapstoneTeamName{		    The Apolloers}
\def \CapstoneTeamNumber{		49}
\def \GroupMemberOne{			Jonathan Ropp}
\def \GroupMemberTwo{			Shannon Sandy}
\def \GroupMemberThree{			Dean Akin}
\def \CapstoneProjectName{		Apollo 11 3D Animation}
\def \CapstoneSponsorCompany{	OMSI}
\def \CapstoneSponsorPersona{	Jim Todd}
\def \CapstoneSponsorPersonb{	Mike Bailey}

% 2. Uncomment the appropriate line below so that the document type works
\def \DocType{		
                %Problem Statement
				%Requirements Document Draft
				%Technology Review
				Design Document
				%Progress Report
				}
			
\newcommand{\NameSigPair}[1]{\par
\makebox[2.75in][r]{#1} \hfil 	\makebox[3.25in]{\makebox[2.25in]{\hrulefill} \hfill		\makebox[.75in]{\hrulefill}}
\par\vspace{-12pt} \textit{\tiny\noindent
\makebox[2.75in]{} \hfil		\makebox[3.25in]{\makebox[2.25in][r]{Signature} \hfill	\makebox[.75in][r]{Date}}}}
% 3. If the document is not to be signed, uncomment the RENEWcommand below
\renewcommand{\NameSigPair}[1]{#1}

%%%%%%%%%%%%%%%%%%%%%%%%%%%%%%%%%%%%%%%
\begin{document}
\begin{titlepage}
    \pagenumbering{gobble}
    \begin{singlespace}
        \hfill 
        % 4. If you have a logo, use this includegraphics command to put it on the coversheet.
        \includegraphics[height=4cm]{OSU_horizontal_2C_O_over_B.eps}   
        \par\vspace{.2in}
        \centering
        \scshape{
            \huge CS Capstone \DocType \par
            {\large\today}\par
            \vspace{.5in}
            \textbf{\Huge\CapstoneProjectName}\par
            \vfill
            {\large Prepared for}\par
            \Huge \CapstoneSponsorCompany\par
            \vspace{5pt}
            {\Large\NameSigPair{\CapstoneSponsorPersona}\par}
            {\Large\NameSigPair{\CapstoneSponsorPersonb}\par}
            {\large Prepared by }\par
            Group\CapstoneTeamNumber\par
            % 5. comment out the line below this one if you do not wish to name your team
            \CapstoneTeamName\par 
            \vspace{5pt}
            {\Large
                \NameSigPair{\GroupMemberOne}\par
                \NameSigPair{\GroupMemberTwo}\par
                \NameSigPair{\GroupMemberThree}\par
            }
            \vspace{20pt}
        }
        \begin{abstract}
        % 6. Fill in your abstract   
    
Next Summer, the Summer of 2019, will be the 50th anniversary of the Apollo 11 Moon Landing and our group, `The Apolloers', is working to create a 3D animation of the mission. This animation is being made for Jim Todd at OMSI so that we can display the animation in the museum as part of their 50th anniversary exhibit. This document will show how we have organized the project and offer different design viewpoints so that stakeholders can see how the project has been structured. 

        \end{abstract}     
    \end{singlespace}
\end{titlepage}
\newpage
\pagenumbering{arabic}
\tableofcontents

% 7. uncomment this (if applicable). Consider adding a page break.
%\listoffigures
%\listoftables
\clearpage

% 8. now you write!

\section{Overview}
       \subsection{Scope}
        This document focuses on the 3D animation for the Apollo 11 Mission that will be created by our group, "The Apolloers" for display at OMSI. The design principals of the project will be outlined here, as well as design elements, such as functionality, usability, and requirements. These principals and elements will outline the deliverables of our project as well as the resources needed to complete it. The context, stakeholders, and intended audience our project will be used in will also be outlined, which will provide supplementary reasoning for our chosen delivarables.  %MORE HERE
        
    \subsection{Purpose}
        The design and structure of this project will be outlined so that readers would be able to understand how the 3D animation is organized from a development viewpoint. In addition to the readers, this document will also serve as guidelines for our group to follow throughout the year. This document will be changed throughout the year as we refine our design. 

    \subsection{Intended Audience}
        This document is intended for the stakeholders of this Apollo 11 3D animation, as well as any interested group that would like to know more about how the animation is structured. Some main takeaways of this document are the design, structure, and connectivity of the components of our animation, so that readers would have a general idea of how to make a similarly structured project. 

\section{Definitions}
%Alphabetize this table!!!
\begin{tabular} {|l|p{13.5cm}|}
\hline
Term & definition \\ \hline
API & An Application Programming Interface is a set of protocols and tools that are used to build a software application. Essentially the `building blocks' that a programmer uses to build an application.  \\ \hline
Apollo 11 Mission & A spaceflight operated by NASA to land the first humans on the Moon, launched July 16th, 1969.  \\ \hline
Graphics Pipeline & A conceptual model that describes the steps that a graphics system needs to perform in order to render a 3D scene. \\ \hline
NASA & The National Aeronautics and Space Administration is a federal agency that focuses on research and development related to air and space.	\\ \hline
OMSI & The Oregon Museum of Science and Industry, located in Portland, Oregon	\\ \hline
OpenAL & An open-source audio library API that is based off of OpenGL so that audio can be implemented into a 3D render in a similar fashion to OpenGL. \\ \hline
OpenGL & An open-source graphics library API that is used to interact with graphics hardware to design 3D renderings.	\\ \hline
SDK & A Software Development Kit is a set of tools that program developers use to write programs for an application. \\ \hline

\end{tabular}

\section{Design Description}

    \subsection{Design Stakeholders}
    The stakeholders for our project are OMSI as well as the audiences who will be viewing our video. The educational and technical quality of our video will reflect on OMSI. Our group will also be using a storyboard created by some of OMSI's staff to base the video's narrative off of, so OMSI will also have influence over how this video is created. The planetarium audiences will also be affected by the technical and educational quality of our video. If the video is more entertaining than educational, then the audience will not understand the impact and ambition of the Apollo 11 mission. If the video is more educational than entertaining, then the audience will be bored and less likely to pay attention to the content of the video.
    
\subsection{Design Views}
    The Apollo 11 animation project can be broken into different views that can all be used to describe the animation. This document will look at the design of the animation through the different viewpoints listed below, and will explain those viewpoints in detail in Section 4. 
    
    \subsection{Design Viewpoints}
    \begin{enumerate}
        \item \textbf{Context}: In what context will this animation be viewed.
        \item \textbf{Composition}: How the animation is separated into different entities.
        \item \textbf{Logical}: What logic is constant throughout the animation, regardless of design decisions.
        \item \textbf{Dependency}: How different entities of the animation depend on others.
        \item \textbf{Information}: What data will be needed and how it will be used in the animation. 
        \item \textbf{Interface}: How developers should correctly use the animation.
        \item \textbf{Resources}: What external entities are needed for the animation. 
    \end{enumerate}
    
    \subsection{Design Elements}
    There are many elements of a 3D animation that all need to work with the other elements to produce a quality project. Programming platforms such as OpenAL and OpenGL will be used for managing the other elements relating to 3D graphics and audio respectively. OpenAL will be used to work with audio files from the transmissions from the mission, as well as various sound effects that will be included to immerse the audience. 
    
    The programming in OpenGL will dictate what 3D objects are in the project, where they are in the scene, and how they look. Many 3D objects in our project will come from an online repository, which we may need to add textures to, which would also be found online. Once the 3D object is in the scene, different lighting features will be programmed in to account for the Sun and that light reflecting off of the other objects. Animating the objects will require use of real physics and orbital mechanics applied to the objects to provide realistic movement. 
    
%   The elements of our project include:
%    - OpenGL
%    - OpenAL
%    - Orbital mechanics
%    - 3D models of .....
%    - Audio files of the mission transmissions and sound effects
%    - Textures for our 3D models
%    - Lighting 
    \subsection{Design Overlays} %Any additional information on certain viewpoints.
	\subsubsection{Textures and 3D Models}
	Most of these textures will come from the NASA websites, which are readily available for everyone to use. Do note that these textures are high poly textures, and are rather accurate as they can be for just being a picture. The 3D model of the Saturn 5 will come from somewhere else, probably from some model sharing site, there is millions of models made within the modeling community, making it highly probable that someone has already made the Saturn 5 3D model. If we cannot find a low poly 3D model of the Saturn 5, we must settle for a high poly model, or maybe will have to make one ourselves. If we are left with a high poly 3D model, we must separate the more fine details from the model in attempt to reduce increase in our polygon budget. Our last choice is to make it ourselves, in this case the project will take more time due to more allocation of time to finish the 3D model.
	\subsubsection{The Flight Path}
	In terms of the flight path, optimally we would want a dataset of the whole flight and back, it is very easy to convert flight data into a virtual path instead of calculating and predicting the path through the use of initial and final variables. If it turns out that we cannot find any dataset for the flight, we can calculate the flight path in a similar amount of time, but will require us to sit down and research orbital mechanics. 
	\subsubsection{Planetarium SDK}
	One of our biggest stretch goals is to manage to pipe in our animation into one of their planetariums. We suspect that their planetariums use some type of SDK to interpolate subsections or FOV's into smaller chunks, we just don't have access to their SDK. The SDK is probably within the hands of the company that built the planetarium for OMSI, and will probably need some negotiation to get our hands on it. 
    \subsection{Design Rationale}
	The basic rationale of our project is that setting the scene correctly is the most important factor in making this animation the best it can be. The Earth, Moon, and Sun need to be in the right position and the camera (or Eye) needs to be in the right position to reflect that. The lighting from the sun needs to be accurate as it reflects and refracts off the lunar capsule, moon, and Earth. The small details matter the most, such as the Earth having its iridescent glow thats given off by its atmosphere, or both planetoids having a dark and a light side. All of these combined truly create the scene, the flight path is the simpliest part, but making it feel like the Apollo 11 mission is the hardest.
    \subsection{Design Languages}



\section{Design Viewpoints}

    \subsection{Context Viewpoint}
    %This will be an OMSI presentation, viewers would be visitors, etc.
    %careful about overlap with Interface viewpoint
    
    This animation will be viewed primarily by OMSI visitors. This audience can range from young school children on field trips, to very technical industry leaders. We will want the user of this program to be able to change the viewpoint of the scene, regardless of if the user is a visitor or a staff member running the program.  
    
    \subsection{Composition Viewpoint}
    %How the project is split into different parts (Audio, modeling, textures, objects, etc.)
    
    Our project will consist of eleven main scenes of the Apollo 11 mission: launch, to orbit, to trans-lunar injection, to moon, to lunar orbit, to landing, to re-launch, to lunar orbit, to trans-Earth injection, to Earth, to splashdown. Then, within each of these scenes, there will be many different parts that make the scene: 3D objects, animation, audio, textures, lighting, orbital mechanics acting on objects, and possibly captioning for the audience. While there are not direct relationships between all these parts, they all add to the overall detail and quality of the animation. 
    
    \subsection{Logical Viewpoint}

    Most of our logic will be based on how far along the mission we are in the animation. We will have a flight path and as we make progress along that path, different actions will happen at different points, such as turning rockets on/off, dropping booster rockets, starting to land, etc. 
    %What is the flight path? An array of vertices? A function?
    
    \subsection{Dependency Viewpoint}

    Some texturing will need to change depending on where the viewpoint is. A main example is that we don't want to use incredibly detailed textures for the Moon when the rocket has barely left Earth, and vice versa. We can have the textures change depending on distance, but this change should not be noticeable to the viewer.
    
    Our audio triggers will be dependant on what point in the mission the video is at. For example, the audio for the Saturn 5 rockets will be triggered during the launch phase of the mission. The volume of the audio will also depend on what viewpoint is currently being used. If the viewpoint is outside of the Apollo 11, the conversations between Tranquility Base and the Apollo 11 will be at a lower volume than if the viewpoint was inside of the Apollo 11.
    
    \subsection{Information Viewpoint}
    %The physics, math, numbers, etc.
    For our project, the information we will need includes the flight path of the Apollo 11 as well as orbital mechanics for the Apollo 11 and solar system. The moon and Apollo 11 will follow orbital mechanics because it is integral to the flight path of the Apollo 11. Our group needs to accurately recreate the flight path in order to make our video as realistic as possible. If the flight path were incorrect, audience members would be wrongly informed about the Apollo 11 mission.

    \subsection{Interface Viewpoint}
       
       The 3D animation will be compiled into an executable file (.exe) so that the user can open the whole animation from one file. Then, the user will be able to start and stop the animation using a keyboard, and change their viewpoint by clicking and dragging their mouse, or choose set viewpoints using the keyboard. If the user does not want to change the viewpoint, the viewpoint will be set to default viewing positions that our group has chosen. The user will be able to view the video from outside of the spacecraft, inside the spacecraft, on the moon, and on Earth. As the animation progresses, there will be functionality for the user to step back or forward through the animation, and possibly jump to pre-defined points in the flight path.
    
    \subsection{Resources Viewpoint}
    To obtain the audio of the transmissions between the Apollo 11 crew and Tranquility Base, we will use NASA's archive of audio files from the Apollo 11 mission. Using OpenAL, these files will be played approximately at the same stage of the mission as they were recorded in. Our group will not be implementing every audio file in the Apollo 11 audio archive. We will be including audio files that are entertaining and not filled with jargon, so the audience can understand and enjoy the conversations between Tranquility Base and the Apollo 11.
    
    To obtain the audio for the sound effects in our project, our group will download audio files from free online sound effects libraries. We will only need two audio files for sound effects, one for the splash of the splashdown and another for the roar of the Saturn 5 rockets. Like the audio files obtained from NASA's archive, the audio will be implemented in a 3D space using OpenAL. 
    % where are we getting 3-D objects, audio files, etc.

\section{Conclusion}
A 3D animation consists of many different elements that all need to work together based on the implementation in a given API. We will be using OpenGL as our primary API that will manage the main graphical elements of our group's Apollo 11 3D animation. The animation will include all parts of the mission from takeoff to splashdown on Earth with full functioning elements such as 3D objects, texturing on those objects, lighting for the scene, etc. With a completed animation, direct users of the animation will be able to use a mouse and keyboard to navigate through the animation and view the scene through arbitrary viewpoints. Lastly, the main goal of this Apollo 11 animation is to engage audiences at OMSI and encourage curiosity in regards to the vastness of space. 

\end{document}
