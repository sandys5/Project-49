\documentclass[onecolumn, draftclsnofoot,10pt, compsoc]{IEEEtran}
\usepackage{graphicx}
\usepackage{url}
\usepackage{setspace}

\usepackage{geometry}
\geometry{textheight=9.5in, textwidth=7in}

% 1. Fill in these details
\def \CapstoneTeamName{		    The Apolloers}
\def \CapstoneTeamNumber{		49}
\def \GroupMemberOne{			Jonathan Ropp}
\def \GroupMemberTwo{			Shannon Sandy}
\def \GroupMemberThree{			Dean Akin}
\def \CapstoneProjectName{		Apollo 11 3D Animation}
\def \CapstoneSponsorCompany{	OMSI}
\def \CapstoneSponsorPersona{	Jim Todd}
\def \CapstoneSponsorPersonb{	Mike Bailey}

% 2. Uncomment the appropriate line below so that the document type works
\def \DocType{		
                %Problem Statement
				%Requirements Document Draft
				%Technology Review
				Design Document
				%Progress Report
				}
			
\newcommand{\NameSigPair}[1]{\par
\makebox[2.75in][r]{#1} \hfil 	\makebox[3.25in]{\makebox[2.25in]{\hrulefill} \hfill		\makebox[.75in]{\hrulefill}}
\par\vspace{-12pt} \textit{\tiny\noindent
\makebox[2.75in]{} \hfil		\makebox[3.25in]{\makebox[2.25in][r]{Signature} \hfill	\makebox[.75in][r]{Date}}}}
% 3. If the document is not to be signed, uncomment the RENEWcommand below
\renewcommand{\NameSigPair}[1]{#1}

%%%%%%%%%%%%%%%%%%%%%%%%%%%%%%%%%%%%%%%
\begin{document}
\begin{titlepage}
    \pagenumbering{gobble}
    \begin{singlespace}
        \hfill 
        % 4. If you have a logo, use this includegraphics command to put it on the coversheet.
        \includegraphics[height=4cm]{OSU_horizontal_2C_O_over_B.eps}   
        \par\vspace{.2in}
        \centering
        \scshape{
            \huge CS Capstone \DocType \par
            {\large\today}\par
            \vspace{.5in}
            \textbf{\Huge\CapstoneProjectName}\par
            \vfill
            {\large Prepared for}\par
            \Huge \CapstoneSponsorCompany\par
            \vspace{5pt}
            {\Large\NameSigPair{\CapstoneSponsorPersona}\par}
            {\Large\NameSigPair{\CapstoneSponsorPersonb}\par}
            {\large Prepared by }\par
            Group\CapstoneTeamNumber\par
            % 5. comment out the line below this one if you do not wish to name your team
            \CapstoneTeamName\par 
            \vspace{5pt}
            {\Large
                \NameSigPair{\GroupMemberOne}\par
                \NameSigPair{\GroupMemberTwo}\par
                \NameSigPair{\GroupMemberThree}\par
            }
            \vspace{20pt}
        }
        \begin{abstract}
        % 6. Fill in your abstract   
        %NOTE: this is the same as Rquirement doc, lets change this
      Our group, The Apolloers, is working with Mike Bailey to create a 3D animation about the Apollo 11 Moon Landing. This animation will be put on display in OMSI during the Summer of 2019 for the 50th anniversary of the Apollo 11 mission. All parts of the mission will be included, from Earth Launch to Earth landing, and all sections in between. This document breaks the project into requirements that we will use to guide our project through the development process. 
        \end{abstract}     
    \end{singlespace}
\end{titlepage}
\newpage
\pagenumbering{arabic}
\tableofcontents

% 7. uncomment this (if applicable). Consider adding a page break.
%\listoffigures
%\listoftables
\clearpage

% 8. now you write!

\section{Overview}
       \subsection{Scope}
        This document focuses on the 3D animation for the Apollo 11 Mission that will be created by our group, "The Apolloers" for display at OMSI. The design principals of the project will be outlined here, as well as design elements, such as functionality, usability, requirements, and more. 
        
    \subsection{Purpose}
        The design of our 3D animation will be outlined throughout this document. While this document will be updated throughout the year, it will start as an outline of how we will be implementing the 3D animation throughout the year. 

    \subsection{Intended Audience}
        This document is for people that are looking for a technical outline of how our 3D animation will be implemented. Some main takeaways of this document are the design, structure, and connectivity of the components of our project.

\section{Definitions}
%Make this a list, check with Richard about using a table
-OMSI: Oregon Museum of Science and Industry
\section{Conceptual Model}
%Mini-requirements doc
    \subsection{}

\section{Design Description}

    \subsection{Design Stakeholders}
    The stakeholders for our project are OMSI as well as the audiences who will be viewing our video. The educational and technical quality of our video will reflect on OMSI. Our group will also be using a storyboard created by some of OMSI's staff to base the video's narrative off of, so OMSI will also have influence over how this video is created. The planetarium audiences will also be affected by the technical and educational quality of our video. If the video is more entertaining than educational, then the audience will not understand the impact and ambition of the Apollo 11 mission. If the video is more educational than entertaining, then the audience will be bored and less likely to pay attention to the content of the video.
    \subsection{Design Views}
    \subsection{Design Viewpoints}
    \subsection{Design Elements}
     The elements of our project include:
    - OpenGL
    - OpenAL
    - Orbital mechanics
    - 3D models of .....
    - Audio files of the mission transmissions and sound effects
    - Textures for our 3D models
    - Lighting 
    \subsection{Design Overlays}
    \subsection{Design Rationale}
    \subsection{Design Languages}


% Richard, Do we need to do ALL viewpoints, or is what we have ok?
\section{Design Viewpoints}

    \subsection{Context Viewpoint}
    %This will be an OMSI presentation, viewers would be visitors, etc.
    %careful about overlap with Interface viewpoint
    
    This animation will be viewed primarily by OMSI visitors. This audience can range from young school children on field trips, to very technical industry leaders. We will want the user of this program to be able to change the viewpoint of the scene, regardless of if the user is a visitor or a staff member running the program.  
    
    \subsection{Composition Viewpoint}
    %How the project is split into different parts (Audio, modeling, textures, objects, etc.)
    
    Our project will consist of eleven main scenes of the Apollo 11 mission: launch, to orbit, to trans-lunar injection, to moon, to lunar orbit, to landing, to re-launch, to lunar orbit, to trans-Earth injection, to Earth, to splashdown. Then, within each of these scenes, there will be many different parts that make the scene: 3D objects, animation, audio, textures, lighting, orbital mechanics acting on objects, and possibly captioning for the audience. While there are not direct relationships between all these parts, they all add to the overall detail and quality of the animation. 
    
    \subsection{Logical Viewpoint}
    %Classes we use, big libraries, etc.
    
    \subsection{Dependency Viewpoint}
    %What we are using? OpenGL
    
    \subsection{Information Viewpoint}
    %The physics, math, numbers, etc.

    \subsection{Interface Viewpoint}
    %How we want the end-product to work
        % .exe file? Blender? Unity?
        
    \subsection{Interaction Viewpoint}
    For our project, the audience will have the option to change the viewpoint of the scene. If the audience does not want to change the viewpoint, the viewpoint will stay at the default viewing positions that our group has chosen. If the audience wishes to change the viewpoint, they will be able to press keys on a keyboard to switch between viewpoints. The audience will be able to view the video from outside of the spacecraft, inside the spacecraft, on the moon, and on Earth.
    % People can just watch, or use mouse/keyboard to move viewpoint
    % What's the mouse/keyboard going to do?
    
    \subsection{Resources Viewpoint}
    To obtain the audio of the transmissions between the Apollo 11 crew and Tranquility Base, we will use NASA's archive of audio files from the Apollo 11 mission.
    
    To obtain the audio for the sound effects in our project, our group will download audio files from free online sound effects libraries.
    % where are we getting 3-D objects, audio files, etc.


\section{Conclusion}


\end{document}
