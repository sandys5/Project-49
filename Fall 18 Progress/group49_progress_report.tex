\documentclass[onecolumn, draftclsnofoot,10pt, compsoc]{IEEEtran}
\usepackage{graphicx}
\usepackage{url}
\usepackage{setspace}

\usepackage{geometry}
\geometry{textheight=9.5in, textwidth=7in}

% 1. Fill in these details
\def \CapstoneTeamName{		    The Apolloers}
\def \CapstoneTeamNumber{		49}
\def \GroupMemberOne{			Jonathan Ropp}
\def \GroupMemberTwo{			Shannon Sandy}
\def \GroupMemberThree{			Dean Akin}
\def \CapstoneProjectName{		Apollo 11 3D Animation}
\def \CapstoneSponsorCompany{	OMSI}
\def \CapstoneSponsorPersona{	Jim Todd}
\def \CapstoneSponsorPersonb{	Mike Bailey}

% 2. Uncomment the appropriate line below so that the document type works
\def \DocType{		
                %Problem Statement
				%Requirements Document Draft
				%Technology Review
				%Design Document
				Progress Report
				}
			
\newcommand{\NameSigPair}[1]{\par
\makebox[2.75in][r]{#1} \hfil 	\makebox[3.25in]{\makebox[2.25in]{\hrulefill} \hfill		\makebox[.75in]{\hrulefill}}
\par\vspace{-12pt} \textit{\tiny\noindent
\makebox[2.75in]{} \hfil		\makebox[3.25in]{\makebox[2.25in][r]{Signature} \hfill	\makebox[.75in][r]{Date}}}}
% 3. If the document is not to be signed, uncomment the RENEWcommand below
\renewcommand{\NameSigPair}[1]{#1}

%%%%%%%%%%%%%%%%%%%%%%%%%%%%%%%%%%%%%%%
\begin{document}
\begin{titlepage}
    \pagenumbering{gobble}
    \begin{singlespace}
        \hfill 
        % 4. If you have a logo, use this includegraphics command to put it on the coversheet.
        \includegraphics[height=4cm]{OSU_horizontal_2C_O_over_B.eps}   
        \par\vspace{.2in}
        \centering
        \scshape{
            \huge CS Capstone \DocType \par
            {\large\today}\par
            \vspace{.5in}
            \textbf{\Huge\CapstoneProjectName}\par
            \vfill
            {\large Prepared for}\par
            \Huge \CapstoneSponsorCompany\par
            \vspace{5pt}
            {\Large\NameSigPair{\CapstoneSponsorPersona}\par}
            {\Large\NameSigPair{\CapstoneSponsorPersonb}\par}
            {\large Prepared by }\par
            Group\CapstoneTeamNumber\par
            % 5. comment out the line below this one if you do not wish to name your team
            \CapstoneTeamName\par 
            \vspace{5pt}
            {\Large
                \NameSigPair{\GroupMemberOne}\par
                \NameSigPair{\GroupMemberTwo}\par
                \NameSigPair{\GroupMemberThree}\par
            }
            \vspace{20pt}
        }
        \begin{abstract}
        % 6. Fill in your abstract   
    Our group, `The Apolloers', has been working to create a 3D animation of the Apollo 11 Moon Space Mission and this document serves as a progress report for Fall 2018, our first term of work. So far, most of our work has focused on documentation and preparing for development, which will start after the end of this term. Two portions of the animation will be completed as a final project for our Into to Computer Graphics course, and will allow for a smoother transition into the project's true development cycle. 

        \end{abstract}     
    \end{singlespace}
\end{titlepage}
\newpage
\pagenumbering{arabic}
\tableofcontents

% 7. uncomment this (if applicable). Consider adding a page break.
%\listoffigures
%\listoftables
\clearpage

% 8. now you write!

\section{Project Purpose}
Next Summer, July 2019, will mark the 50th anniversary of the Apollo 11 Mission. This mission, operated by The National Aeronautics and Space Administration (NASA), made history by making Neil Armstrong the first human to set foot on the Moon. Our project, proposed by The Oregon Museum of Science and Industry (OMSI), is to create a 3D animation of the entire Apollo 11 Mission to be put on display in OMSI's commemorative display next Summer. The purpose of this animation is to give viewers an accurate depiction of the Apollo 11 mission with all its complexity, as well being a tool to learn about space travel and to inspire curiosity. 

\section{Goals}
Our animation, which will primarily be made using OpenGL, will need to be fully functional before the OSU Engineering Expo on May 18th, 2019. This means that the entire mission is animated and allows users to take control of the viewpoints and time of the mission. If possible, the animation will also need to be able to be shown in the OMSI planetarium, but it is uncertain if our group will gain access to the development tools necessary to show the animation using the 10-projector setup. 
\newline
\newline
Another goal is to include audio along with the animation, particularly the mission transmissions between the the astronauts and Earth. Building off of this, captions may also be included for the viewers to know what is being said, notably depending on the quality of audio that is able to be obtained. Our overarching goal for the viewers is to make sure that the wide variety of audiences that OMSI draws will all be able to view and interact with the 3D animation with ease. Simple user tests may be organized later to ensure this goal is met. 


\section{Current Status}
While our group will not officially start development until the end of this term, the final project for the Introduction to Computer Graphics course that our group is in allowed us to choose our implementation. Two of our members decided to use this opportunity to start creating scenes for the Apollo 11 3D animation, namely the overall scene (Earth, Moon, background, etc.) and the final splashdown. These will provide great starting points for when our group truly starts development. Lastly, to finish out this term, we plan to update all current documentation and make sure that our group repository contains all our work and is ready for when deveopment starts. 

\section{Issues}
This term, our group's most prevalent problem was how to document our project according to the guidelines set. We often found ourselves having issues following IEEE standards because our project did not seem to align well with those standards. Also, in terms of a 3D animation, most of the organization that needs to be done is included when deciding what to include in the animation, but in our case, our animation has already been laid out since it will follow the Apollo 11 Mission. This all culminated into frustrations with this term's documentation, but with help from our TA, Richard Cunard, and reviewing how previous groups approached similar issues helped to guide our writing. 

\section{Retrospective}

\begin{tabular}{|p{0.3\linewidth}|p{0.3\linewidth}|p{0.3\linewidth}|}
\hline
     Positives & Deltas & Actions  \\
    \hline
    
    
    
    \textbf{Resources:} Most of the resources we need are readily available to us and we have composed a list of requirements for the 3D animation to guide us through development. & Our group does not yet have access to the OMSI SDK that will be required for display in the planetarium. & Mike Bailey will be trying to contact the company that built the planetarium for OMSI in an attempt to gain access to the SDK.\\ 
     \hline
	\textbf{Pre-Development:} We will be able to complete two scenes of the animation as final projcets for our Intro to Computer Graphics course, meaning that we will have a head start for development & A large portion of code will need to be set up as a foundation for the environment of the project. & We are all going to be working on drafting scenes and the code foundation throughout the Winter break progressively.\\
     \hline
     \textbf{Computer Graphics:} Our group has learned the basics of computer graphics through Mike Bailey's course this term. We all have the knowledge for implementation, and can more easily research the area of computer graphics as needed. & In order to keep our polygon count low, we will need to have extensive knowledge of shaders create detailed scenes. & Dean Akin and Shannon Sandy will be taking an additional shaders course with Mike Bailey during Winter term to gain more knowledge of the subject.\\
     \hline
     \textbf{Documentation:} Throughout the term, we have outlined how our project will be made and with what resources. We have received feedback from many sources to continually improve our work. & We need to send our documents to our client to receive verification that we are on the right track and so that we can begin development. & We are currently planning a trip to OMSI with Mike Bailey to meet Jim Todd in person before the end of Winter break. This will allow us to discuss the logistics of the project with Jim as well as gaining an understanding of what kind of storyboard is expected.\\
     \hline
     \textbf{Technologies:} Each member of our group reviewed a technology that would be implemented in our animation: 3D models, audio, and the API to be used. We explored many more options and became more confident about our choices. & There are some technology choices that will not be able to be made until later in development, such as how to best compile our project to be the most accessible at OMSI. & Our trip to OMSI will be a great chance to ask these questions, as well as seeing exactly what options we have to work with. \\ 
     \hline
     
\end{tabular}

\end{document}
