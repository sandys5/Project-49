\documentclass[onecolumn, draftclsnofoot,10pt, compsoc]{IEEEtran}
\usepackage{graphicx}
\usepackage{url}
\usepackage{setspace}

\usepackage{geometry}
\geometry{textheight=9.5in, textwidth=7in}

% 1. Fill in these details
\def \CapstoneTeamName{		    The Apolloers}
\def \CapstoneTeamNumber{		49}
\def \GroupMemberOne{			Jonathan Ropp}
\def \GroupMemberTwo{			Shannon Sandy}
\def \GroupMemberThree{			Dean Akin}
\def \CapstoneProjectName{		Apollo 11 3D Animation}
\def \CapstoneSponsorCompany{	OMSI}
\def \CapstoneSponsorPersona{	Jim Todd}
\def \CapstoneSponsorPersonb{	Mike Bailey}

% 2. Uncomment the appropriate line below so that the document type works
\def \DocType{		
                %Problem Statement
				%Requirements Document Draft
				%Technology Review
				%Design Document
				Progress Report
				}
			
\newcommand{\NameSigPair}[1]{\par
\makebox[2.75in][r]{#1} \hfil 	\makebox[3.25in]{\makebox[2.25in]{\hrulefill} \hfill		\makebox[.75in]{\hrulefill}}
\par\vspace{-12pt} \textit{\tiny\noindent
\makebox[2.75in]{} \hfil		\makebox[3.25in]{\makebox[2.25in][r]{Signature} \hfill	\makebox[.75in][r]{Date}}}}
% 3. If the document is not to be signed, uncomment the RENEWcommand below
\renewcommand{\NameSigPair}[1]{#1}

%%%%%%%%%%%%%%%%%%%%%%%%%%%%%%%%%%%%%%%
\begin{document}
\begin{titlepage}
    \pagenumbering{gobble}
    \begin{singlespace}
        \hfill 
        % 4. If you have a logo, use this includegraphics command to put it on the coversheet.
        \includegraphics[height=4cm]{OSU_horizontal_2C_O_over_B.eps}   
        \par\vspace{.2in}
        \centering
        \scshape{
            \huge CS Capstone \DocType \par
            {\large\today}\par
            \vspace{.5in}
            \textbf{\Huge\CapstoneProjectName}\par
            \vfill
            {\large Prepared for}\par
            \Huge \CapstoneSponsorCompany\par
            \vspace{5pt}
            {\Large\NameSigPair{\CapstoneSponsorPersona}\par}
            {\Large\NameSigPair{\CapstoneSponsorPersonb}\par}
            {\large Prepared by }\par
            Group\CapstoneTeamNumber\par
            % 5. comment out the line below this one if you do not wish to name your team
            \CapstoneTeamName\par 
            \vspace{5pt}
            {\Large
                \NameSigPair{\GroupMemberOne}\par
                \NameSigPair{\GroupMemberTwo}\par
                \NameSigPair{\GroupMemberThree}\par
            }
            \vspace{20pt}
        }
        \begin{abstract}
        % 6. Fill in your abstract   
    Our group, `The Apolloers', have been working on a 3D animation of the Apollo 11 Moon Space Mission and this document provides a progress report for our first term of work. So far, most of our work has focused on documentation and preparing for development, which will start after this term, Fall 2018. % add more


        \end{abstract}     
    \end{singlespace}
\end{titlepage}
\newpage
\pagenumbering{arabic}
\tableofcontents

% 7. uncomment this (if applicable). Consider adding a page break.
%\listoffigures
%\listoftables
\clearpage

% 8. now you write!

\section{Project Purpose}
Next Summer, July 2019, will mark the 50th anniversary of the Apollo 11 Mission. This mission, operated by The National Aeronautics and Space Administration (NASA), made history by making Neil Armstrong the first human to set foot on the Moon. Our project, proposed by The Oregon Museum of Science and Industry (OMSI), is to create a 3D animation of the entire Apollo 11 Mission to be put on display in OMSI's commemorative display next Summer. Users of the animation will be able to control the view of the mission during the animation and step through the animation at their discretion. The purpose of this animation is to give viewers an accurate depiction of the Apollo 11 mission with all its complexity, as well being a tool to learn about space travel and to inspire curiosity. 

\section{Goals}


\section{Current Status}


\section{Issues}


\section{Retrospective}

\begin{tabular}{|l|l|l|l|}
\hline
    Task & Positives & Deltas & Actions  \\
    
    \hline
     
\end{tabular}

\end{document}
